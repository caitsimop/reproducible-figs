%%%%%% NOTES %%%%%%%
% Uncomment for layout with notes
%\documentclass[handout]{beamer}
%\usepackage{pgfpages}
%\pgfpagesuselayout{4 on 1}[a4paper,border shrink=5mm]

\documentclass[table, svgnames,dvipsnames,14pt, aspectratio=169]{beamer}

%\usepackage{multirow}
%\usepackage{array}
\usepackage{booktabs}
\usepackage{graphicx}
\usepackage{fontawesome} %% for social media icons
%\usepackage{adjustbox}
\usepackage{relsize}
\usepackage[font=footnotesize, labelformat=empty, justification=raggedright]{caption}
%\usepackage{booktabs}
\usepackage{xspace}
%\usepackage{threeparttable}
%\usepackage{pgfgantt}
\usepackage{textpos}
\usepackage{fancybox}
%\usepackage{appendixnumberbeamer}
%\setbeamertemplate{page number in head/foot}[appendixframenumber]
\usepackage[framemethod=tikz]{mdframed}
%%$\usepackage{enumitem}
%%%Font change
\usepackage[T1]{fontenc}
\usepackage{cmbright} %%maybe too comic sans-y?
%
%\usepackage[latin1]{inputenc}
%%\usepackage{color}
%%\usepackage{appendixnumberbeamer}
%%\hypersetup{colorlinks,linkcolor=,urlcolor=blue}
%%\usepackage{adjustbox}
\usepackage{animate}
%\usepackage{glossaries}
\usepackage{tikz}
\usetikzlibrary{arrows,positioning, shapes.symbols,shapes.callouts,patterns, shapes, backgrounds, arrows.meta, decorations.pathreplacing} %,shapes.emoticon}
%\usepackage[export]{adjustbox}
%
%\usepackage{hyperref}
% Preamble!! :)


%\usepackage{pgfgantt}
\setbeamercovered{dynamic}


%\usepackage{appendixnumberbeamer} % use this for extra slides (appendix)

%\usetheme{bjeldbak}

\setbeamertemplate{note page}[plain]
% No navigation
\setbeamertemplate{frametitle}
  {\begin{centering}\smallskip
   \insertframetitle\par
   \smallskip\end{centering}}
\setbeamertemplate{itemize item}{$\bullet$}
\setbeamertemplate{navigation symbols}{}



% Colours
%\definecolor{Berry}{RGB}{194, 52, 102}
%\definecolor{Forest}{RGB}{33, 106, 18}
%\definecolor{Orange}{RGB}{222, 93, 59}
%\definecolor{DarkCharcoal}{HTML}{4D4944}
%\colorlet{DarkForest}{Forest!80!black}
%\colorlet{DarkOrange}{Orange!80!black}
%\colorlet{Charcoal}{DarkCharcoal!85!white}
%\colorlet{DarkBerry}{Berry!60!black}
% New Colours :)
%\definecolor{Navy}{RGB}{3, 1, 44}
%\definecolor{Purple}{RGB}{65, 34, 52}
%\definecolor{PrettyGreen}{RGB}{47, 86, 75}
%\definecolor{Pink}{RGB}{255, 105, 120}
%\definecolor{OffWhite}{RGB}{255, 234, 208}
%\colorlet{WhiteOffWhite}{OffWhite!95!white}
%\definecolor{GreyWhite}{RGB}{255,234,208}
%\definecolor{DarkPurple}{RGB}{51, 41, 56}
%\definecolor{LightishPurple}{RGB}{109, 64, 87}
\definecolor{Pinky}{RGB}{229, 104, 97}
%\definecolor{Robin}{RGB}{172, 229, 223}
%\colorlet{DarkRobin}{Robin!80!black}
%\colorlet{LightNavy}{Navy!50!white}

% New color scheme maybe!
\definecolor{DKoamaru}{HTML}{2B4162}
\definecolor{Peach}{HTML}{FFB4A2}
\definecolor{DSparkle}{HTML}{385F71}
\definecolor{Wine}{HTML}{381D2A}
\definecolor{PurpNav}{HTML}{4D5382}
\definecolor{Smoke}{HTML}{F5F0F6}
\definecolor{Carrot}{HTML}{F9A03F}
\definecolor{TerraCotta}{HTML}{E87461}
\definecolor{Eggplant}{HTML}{5E4352}
\definecolor{paleyellow}{HTML}{FDEF8D}
\definecolor{paleblue}{HTML}{769FAE}
\definecolor{palepurple}{HTML}{C0A9B0}
\definecolor{cadet}{HTML}{5CA4A9}
%\definecolor{orangeyellow}{HTML}{FFD166}
%\definecolor{orangeyellow}{HTML}{F39237}
\definecolor{orangeyellow}{HTML}{E9B44C}
%\definecolor{orangeyellow}{HTML}{F5853F}

%%%%%%% defence colours
\definecolor{newgreen}{HTML}{628062}
\definecolor{outerspace}{HTML}{444554}
\definecolor{pinkaccent}{HTML}{CC7E85}
\definecolor{darkfill}{HTML}{764248}
\definecolor{tealblue}{HTML}{33658A}

\definecolor{lightblue}{HTML}{AED9E0}
\definecolor{darkblue}{HTML}{506366}
\definecolor{lightpink}{HTML}{FE9992}
\definecolor{darkpink}{HTML}{A2625D}


%%%%%%% new palette


\definecolor{midnightpurp}{HTML}{631D76}
\definecolor{magentahaze}{HTML}{9E4770}
\definecolor{coralreef}{HTML}{FF7E6B}
\definecolor{lapis}{HTML}{2274A5}
\definecolor{warmblack}{HTML}{003844}



%%%%%%%%%%
% Use the colors: %
%%%%%%%%%%

%\setbeamercolor{title}{fg=DarkForest}
%\setbeamercolor{frametitle}{fg=DarkBerry}
%\setbeamercolor{normal text}{fg=DarkCharcoal}
%\setbeamercolor{block title}{fg=black,bg=Forest!25!white}
%\setbeamercolor{block body}{fg=black,bg=Forest!25!white}
%\setbeamercolor{alerted text}{fg=DarkOrange}
%\setbeamercolor{itemize item}{fg=Charcoal}
%\setbeamercolor{framesource}{fg=gray}
%\setbeamercolor{section in toc}{fg=DarkForest}
%\setbeamercolor{footnote}{fg=DarkCharcoal}
%\setbeamercolor{footnote mark}{fg=DarkCharcoal}
%\setbeamercolor{myfootlinetext}{fg=DarkCharcoal}
%\setbeamertemplate{itemize subitem}{\color{DarkForest}$\blacktriangleright$} % change sublist bullet to DarkForest


% New colours!
\setbeamercolor{title}{fg=warmblack}
\setbeamercolor{frametitle}{fg=white, bg=lapis}
\setbeamercolor{normal text}{fg=warmblack}
\setbeamercolor{block title}{fg=white,bg=magentahaze}
\setbeamercolor{block body}{fg=outerspace,bg=magentahaze!25!white}
\setbeamercolor{alerted text}{fg=magentahaze}
\setbeamerfont{alerted text}{series=\bfseries}
\setbeamercolor{itemize item}{fg=lapis}
\setbeamercolor{framesource}{fg=lapis}
\setbeamercolor{section in toc}{fg=midnightpurp}
\setbeamercolor{footnote}{fg=warmblack}
\setbeamercolor{footnote mark}{fg=warmblack}
\setbeamercolor{myfootlinetext}{fg=warmblack}
\setbeamertemplate{itemize subitem}{\color{lapis}$\blacktriangleright$}
\setbeamertemplate{itemize subsubitem}{\color{lapis}$\blacktriangleright$}
\setbeamertemplate{itemize item}[circle]
\setbeamercolor*{enumerate item}{fg=warmblack}
\setbeamercolor*{enumerate subitem}{fg=warmblack}
\setbeamercolor*{enumerate subsubitem}{fg=warmblack}
\setbeamerfont{frametitle}{size=\Large} % Larger slide title font
\setbeamertemplate{section in toc}[sections numbered] % numbered TOC sections
\setbeamertemplate{frametitle}[default][left] %centeredframe tittle
\setbeamertemplate{footline}[frame number]
\setbeamerfont{footnote}{size=\tiny} % set footnote font size smaller
\setbeamercolor{button}{bg=magentahaze,fg=white}



%\setbeamercolor{title}{fg=outerspace}
%\setbeamercolor{frametitle}{fg=white, bg=newgreen}
%\setbeamercolor{normal text}{fg=Wine}
%\setbeamercolor{block title}{fg=white,bg=lightpink}
%\setbeamercolor{block body}{fg=outerspace,bg=lightpink!25!white}
%\setbeamercolor{alerted text}{fg=lightpink}
%\setbeamerfont{alerted text}{series=\bfseries}
%\setbeamercolor{itemize item}{fg=darkfill}
%\setbeamercolor{framesource}{fg=lightpink}
%\setbeamercolor{section in toc}{fg=darkfill}
%\setbeamercolor{footnote}{fg=darkfill}
%\setbeamercolor{footnote mark}{fg=darkfill}
%\setbeamercolor{myfootlinetext}{fg=darkfill}
%\setbeamertemplate{itemize subitem}{\color{darkfill}$\blacktriangleright$}
%\setbeamertemplate{itemize subsubitem}{\color{darkfill}$\blacktriangleright$}
%\setbeamertemplate{itemize item}[circle]
%\setbeamercolor*{enumerate item}{fg=darkfill}
%\setbeamercolor*{enumerate subitem}{fg=darkfill}
%\setbeamercolor*{enumerate subsubitem}{fg=darkfill}
%\setbeamerfont{frametitle}{size=\Large} % Larger slide title font
%\setbeamertemplate{section in toc}[sections numbered] % numbered TOC sections
%\setbeamertemplate{frametitle}[default][left] %centeredframe tittle
%\setbeamertemplate{footline}[frame number]
%\setbeamerfont{footnote}{size=\tiny} % set footnote font size smaller
%\setbeamercolor{button}{bg=darkfill,fg=white}


%\setbeamertemplate{footline[textline]}{%
%\hfill\hspace{0.5cm} \scriptsize\sf\color{black!60}\insertframenumber{}/\inserttotalframenumber}

%\setbeamertemplate{footline}[text line]{%
%  \hfill\strut{%
%      \scriptsize\sf\color{black!60}%
%     \quad\insertframenumber
 %   }%
 %   \hfill
%}

% Macros
\newcommand{\bi}{\begin{itemize}}
\newcommand{\ei}{\end{itemize}}
\newcommand{\be}{\begin{enumerate}}
\newcommand{\ee}{\end{enumerate}}
%\newcommand{\esa}{\textit{E.\,salsugineum}}
\newcommand{\esa}{\textit{Eutrema}}
% source at bottom left of slide
\newcommand{\sourceleft}[1]{\begin{textblock*}{4cm}(0.3cm,8.8cm)
    \begin{beamercolorbox}[ht=0.5cm,left]{framesource}
        \usebeamerfont{framesource}\usebeamercolor[fg]{framesource}
{#1}
    \end{beamercolorbox}
\end{textblock*}}

% source at bottom right of slide
\newcommand{\sourceright}[1]{\begin{textblock*}{4cm}(8.6cm,8.8cm)
    \begin{beamercolorbox}[ht=0.5cm,left]{framesource}
        \usebeamerfont{framesource}\usebeamercolor[fg]{framesource}
{#1}
    \end{beamercolorbox}
\end{textblock*}}

\newcommand{\sourceextremeright}[1]{\begin{textblock*}{4cm}(10.6cm,8.8cm)
    \begin{beamercolorbox}[ht=0.5cm,left]{framesource}
        \usebeamerfont{framesource}\usebeamercolor[fg]{framesource}
{#1}
    \end{beamercolorbox}
\end{textblock*}}

% TOC shows at beginning of section
%\AtBeginSection[]
%{%
%   \begin{frame}
%       \frametitle{Overview}
%       \tableofcontents[currentsection,hideothersubsections]
%   \end{frame}
%}

% To continue enumerate counts on another slide
\newcounter{saveenumi}
\newcommand{\seti}{\setcounter{saveenumi}{\value{enumi}}}
\newcommand{\conti}{\setcounter{enumi}{\value{saveenumi}}}

\resetcounteronoverlays{saveenumi}

% for citations
%\bibliography{/home/caitlin/Documents/compExam/references2}

% set footnote font size small
\renewcommand{\footnotesize}{\scriptsize}
%\newrobustcmd*{\fcite}{\AtNextCite{\renewbibmacro{title}{}\renewbibmacro{in:}{}}\footfullcite}
% change footnote citation color from blue to DarkCharcoal
\newcommand{\cfcite}[1]{\footnote{\citeauthor{#1}, \citeyear{#1}}} 

\setbeamercolor{bibliography entry author}{fg=Wine}

% For layout with notes!
%\pgfpagesuselayout{3 on 1 with notes}[a4paper,border shrink=5mm]



%% "only"strikeout
\usepackage{ulem}
\renewcommand<>{\sout}[1]{
  \only#2{\beameroriginal{\sout}{#1}}
  \invisible#2{#1}
}

\newmdenv[tikzsetting={fill=white,fill opacity=0.85, line width=4pt},backgroundcolor=none,leftmargin=0,rightmargin=0,innertopmargin=4pt]{TitleBox}

%~~~~~~~~~~~~~~~~~~~~~~~~~~~~~~~~~~~~~~~~~~~~~~~~~~~~~~~~~~~~~~~~~~~
\makeatletter
\let\@@magyar@captionfix\relax
\makeatother

\begin{document}
\title{\Large \textbf{Stats tutorial: \\ Machine learning}}
\author{Caitlin Simopoulos \\ \faGithubSquare~ \vspace{.5cm} \faTwitterSquare~ caitsimop}
%\subtitle{Lab meeting}
\date{January 29, 2020}

%~~~~~~~~~~~~~~~~~~~~~~~~~~~~~~~~~~~~~~~~~~~~~~~~~~~~~~~~~~~~~~~~~~~
{  \usebackgroundtemplate{\includegraphics[width=1.0\paperwidth]{figs/ml.jpg}}
\begin{frame}[plain] 

%  \vspace{15em}
\centering
%      \begin{figure}
      \begin{TitleBox}
   
      \vspace{.1cm} \centering \inserttitle
 
      \vspace{.5cm}

          {\normalsize January 29, 2020 \\ Caitlin Simopoulos }
       

      \vspace{0.5cm}

      {\normalsize  csimopou@uottawa.ca \\  \faGithubSquare~ \vspace{.5cm} \faTwitterSquare~ caitsimop} 
%    {\footnotesize \href{http://twitter.com/adaptive_plant}{{\FA \faTwitter} adaptive\_plant}
%    \href{http://www.falsters.net/daniel}{{\FA \faHome} www.falsters.net/daniel}
%    \href{mailto: daniel.falster@mq.edu.au}{{\FA \faEnvelope}  daniel.falster@mq.edu.au}
%    }
   \end{TitleBox}

  \end{frame}
    }
%~~~~~~~~~~~~~~~~~~~~~~~~~~~~~~~~~~~~~~~~~~~~~~~~~~~~~~~~~~~~~~~~~~~

\begin{frame}{Overview}

\be
\item What is machine learning?
\item Examples of ML algorithms, touching on:
    \bi
\item Supervised
\item Unsupervised
\item Deep learning
    \ei
\item Common challenges and limitations of ML
\item How you can use ML in your own work    
\ee

\end{frame}

%~~~~~~~~~~~~~~~~~~~~~~~~~~~~~~~~~~~~~~~~~~~~~~~~~~~~~~~~~~~~~~~~~~~
\begin{frame}{What is machine learning?}

    \only<1>{\centering 
\Large
``Field of study that gives computers the ability to learn without being explicitly programmed.''\\
    \begin{flushright} --Arthur Samuel (1959) \end{flushright}}
     
     \only<2>{\centering 
     \includegraphics[height=.78\textheight]{figs/endofcode.png}

    { \hfill\footnotesize ``Soon we won't program computers, we'll train them like dogs." - Jason Tanz, Wired (2016) \\ 
    \hfill \tiny Art by Edward C. Monaghan  (2016)}
    }

\end{frame}
%~~~~~~~~~~~~~~~~~~~~~~~~~~~~~~~~~~~~~~~~~~~~~~~~~~~~~~~~~~~~~~~~~~~
    \begin{frame}{Artificial Intelligence?}

\centering
        \only<1>{       \includegraphics[height=.9\textheight]{figs/robot.jpg}}
        \only<2>{\includegraphics[height=.8\textheight]{figs/ml_explained.jpg}

        \hfill \footnotesize \url{https://vas3k.com/blog/machine_learning/}}

%Artificial intelligence is the name of a whole knowledge field, similar to biology or chemistry.

\end{frame}
%~~~~~~~~~~~~~~~~~~~~~~~~~~~~~~~~~~~~~~~~~~~~~~~~~~~~~~~~~~~~~~~~~~~
\begin{frame}{Machine learning}
    \only<1>{    \begin{columns}
        \column{.6\textwidth}
        \bi
\item Computers (machines) recognizing patterns in data 
    \bi
\item \textit{i.e.} ``Learn'' from data! 
    \ei
\item Make predictions or group data points
\item Supervised - classification or regression
\item Unsupervised - exploratory/clustering 
\ei    
        \column{.4\textwidth}
        \centering
        \includegraphics[width=\textwidth]{figs/apps.png}
    \end{columns}}

    \only<2>{
        \centering
        \includegraphics[height=.8\textheight]{figs/ml_types.jpg}

 \hfill \footnotesize \url{https://vas3k.com/blog/machine_learning/}
    }
\end{frame}    
%~~~~~~~~~~~~~~~~~~~~~~~~~~~~~~~~~~~~~~~~~~~~~~~~~~~~~~~~~~~~~~~~~~~
\begin{frame}{\normalsize Machine learning we have all used: Linear regression}

    \only<1>{
        \centering
        \includegraphics[height=.9\textheight]{figs/tree_scatter.pdf}
    }
    \only<2>{
        \centering
        \includegraphics[height=.9\textheight]{figs/tree_lm.pdf}
    }
    \only<3>{
        \centering
        \includegraphics[height=.9\textheight]{figs/tree_predict.pdf}

    }

    \only<4>{
        This is a very simple example. In reality there are many other factors that \alert{may} be contributing to the volume of timber.

    \vspace{0.5cm}
     For example:
     \bi
 \item Species of tree
 \item Amount of precipitation 
 \item Temperature
 \item Age of tree
 \item Quality of soil
 \item Presence of pests

     \ei
    }
\end{frame}
%~~~~~~~~~~~~~~~~~~~~~~~~~~~~~~~~~~~~~~~~~~~~~~~~~~~~~~~~~~~~~~~~~~~
\begin{frame}
\centering
\Large

Supervised learning

\end{frame}
%~~~~~~~~~~~~~~~~~~~~~~~~~~~~~~~~~~~~~~~~~~~~~~~~~~~~~~~~~~~~~~~~~~~
\begin{frame}{Supervised learning}
\centering
    \includegraphics[height=.9\textheight]{figs/inputoutput.pdf}

\end{frame}
%~~~~~~~~~~~~~~~~~~~~~~~~~~~~~~~~~~~~~~~~~~~~~~~~~~~~~~~~~~~~~~~~~~~
\begin{frame}{Machine learning lingo\footnote[frame]{\href{https://ml-cheatsheet.readthedocs.io/en/latest/glossary.html}{Click here for more ML terms and their definitions.}}}

\bi
\item \textbf{Feature} - A measurable property, characteristic
\item \textbf{Algorithm} - The model/method used to "learn" how to get from input to output 
\item \textbf{Discrete data} - Categorical data (\textit{e.g.} fur colour)
\item \textbf{Continuous data} - Numerical data that can be any possible number; a numerical measurement (\textit{e.g.} length of tail in cm)
\item \textbf{Training data} - "Known" data that your algorithm learns from
\item \textbf{Classification} - 
\item \textbf{Clustering}
\ei

    \vspace{1cm}

\end{frame}
%~~~~~~~~~~~~~~~~~~~~~~~~~~~~~~~~~~~~~~~~~~~~~~~~~~~~~~~~~~~~~~~~~~~
\begin{frame}{Supervised learning: Decision tree}
\centering
    \includegraphics[height=.9\textheight]{figs/decisiontree.pdf}


\end{frame}    
%~~~~~~~~~~~~~~~~~~~~~~~~~~~~~~~~~~~~~~~~~~~~~~~~~~~~~~~~~~~~~~~~~~~
\begin{frame}{Ensemble learning: Random forest}

    {\centering
    \includegraphics[width=.9\textwidth]{figs/rf.pdf}}

    \vspace{.5cm}
    Many decision trees (to make a \alert{forest}!)
    \bi
\item All trees are ``weak'' learners
\item Trained on bootstrapped training data
    \ei

    Final prediction is make by taking a vote from each tree.

\end{frame}
%~~~~~~~~~~~~~~~~~~~~~~~~~~~~~~~~~~~~~~~~~~~~~~~~~~~~~~~~~~~~~~~~~~~
\begin{frame}{Support vector machines}
\end{frame}
%~~~~~~~~~~~~~~~~~~~~~~~~~~~~~~~~~~~~~~~~~~~~~~~~~~~~~~~~~~~~~~~~~~~
\begin{frame}
\centering
\Large
Unsupervised learning

\end{frame}
%~~~~~~~~~~~~~~~~~~~~~~~~~~~~~~~~~~~~~~~~~~~~~~~~~~~~~~~~~~~~~~~~~~~
\begin{frame}{Unsupervised learning; clustering}
    \begin{columns}
        \column{.5\textwidth}
\centering
        \includegraphics[height=.78\textheight]{figs/unsupervised.pdf}
        \column{.6\textwidth}
\bi
\item Learns from \alert{unlabelled} data (data without ``output'')
\item Clustering is a common technique 
    \bi
        \item Find similarities in the data
        \item ``Each cluster is is distinct..., and the objects within each cluster are broadly similar to each other". \footnote[frame]{\href{https://towardsdatascience.com/https-towardsdatascience-com-hierarchical-clustering-6f3c98c9d0ca}{Link to: Hierarchical clustering clearly explained}}
    \ei
    \ei

    \end{columns}

\end{frame}
%~~~~~~~~~~~~~~~~~~~~~~~~~~~~~~~~~~~~~~~~~~~~~~~~~~~~~~~~~~~~~~~~~~~
\begin{frame}{Hierarchical clustering}

\centering
    \includegraphics[width=\textwidth]{figs/hclust.pdf}
\end{frame}    
%~~~~~~~~~~~~~~~~~~~~~~~~~~~~~~~~~~~~~~~~~~~~~~~~~~~~~~~~~~~~~~~~~~~
\begin{frame}{Evaluating detected clusters}
    \begin{columns}
        \column{.4\textwidth}
\bi
\item ``Cutting tree'' is subjective 
        \bi
    \item Can be by number of expected clusters
    \item Can be by distance between clusters (height in plot)
        \ei
    \item Measure cluster metrics 
            \bi
            \item Silhouette
            \item \sout<2>{Bootstrapping}
             \ei

        \ei

        \column{.6\textwidth}
        \centering
        \includegraphics[height=.4\textheight]{figs/hierbigger.pdf}
\includegraphics[height=.4\textheight]{figs/silhouette.pdf}
    \end{columns}
\end{frame}
%~~~~~~~~~~~~~~~~~~~~~~~~~~~~~~~~~~~~~~~~~~~~~~~~~~~~~~~~~~~~~~~~~~~
%\begin{frame}{Distance measurements}
%
%
%\end{frame}   
%~~~~~~~~~~~~~~~~~~~~~~~~~~~~~~~~~~~~~~~~~~~~~~~~~~~~~~~~~~~~~~~~~~~
\begin{frame}{Principal Component Analysis (PCA)}

    \centering

    \includegraphics[width=.8\textwidth]{figs/pca_visual.png}
   
    \alert{\href{http://setosa.io/ev/principal-component-analysis/}{Click to go to PCA explained visually}}
\end{frame}
%~~~~~~~~~~~~~~~~~~~~~~~~~~~~~~~~~~~~~~~~~~~~~~~~~~~~~~~~~~~~~~~~~~~

\begin{frame}{PCA vs PLS-DA \footnote{\href{https://www.futurelearn.com/courses/metabolomics/0/steps/25039}{Link to: PCA vs PLS-DA info}}}

    \only<1>{    \begin{columns}
        \column{.5\textwidth}
        {\centering\textbf{PCA}}
\bi
\item Unsupervised learning
\item Unbiased
\item Rotates data so that largest differences emphasized
\item Loadings can describe the most influential features that drive PC scores
    \ei
        
        \column{.5\textwidth}
       {\centering\textbf{PLS-DA}}
\bi
\item Supervised learning
\item Explains max. separation between labelled samples %(helpful if differences between classes is smaller than individual variation)
\item Regression is performed to determine which features are important to class membership (labelled samples)
\bi    
\item VIP score
    \ei
\ei
 
    \end{columns}}

    \only<2>{
        \begin{columns}
        \column{.5\textwidth}
        {\centering\textbf{PCA}}
            \centering
            \includegraphics[width=.9\textwidth]{figs/pca_example.pdf}
        \column{.5\textwidth}
       {\centering\textbf{PLS-DA}}
    \end{columns}}
    

%    Partial least squares-discriminant analysis (PLS-DA) is a \alert{supervised} machine learning technique.

%    PCA is an \alert{unsupervised} machine learning technique.


\end{frame}
%~~~~~~~~~~~~~~~~~~~~~~~~~~~~~~~~~~~~~~~~~~~~~~~~~~~~~~~~~~~~~~~~~~~


\begin{frame}{K-means clustering}
%    \only<1>{
%        \centering
%\includegraphics[width=.9\textwidth]{../mcmaster_labMeetings/2017Mar/figs/k-means.png}
%    }
%
%    \only<2>{
\centering
  \animategraphics[loop,controls,height=.75\textheight]{2}{../mcmaster_labMeetings/2017Mar/figs/output/tmp-}{0}{12}

    {\hfill\footnotesize\url{http://www.turingfinance.com/clustering-countries-real-gdp-growth-part2}}
%}


%Initialization, Assignment, and Update). These steps are repeated until either the clustering has converged or the number of iterations has been exceeded a.k.a the computational budget has been exhausted.
%A set of  centroids are randomly initialized in the search space.
% assign each point closest to each centroid
% centroids value is replaced with the mean of every point that has been assigned to it
% New iteration
\end{frame}
%~~~~~~~~~~~~~~~~~~~~~~~~~~~~~~~~~~~~~~~~~~~~~~~~~~~~~~~~~~~~~~~~~~~
\begin{frame}
\centering
\Large
Deep learning
\end{frame}
%~~~~~~~~~~~~~~~~~~~~~~~~~~~~~~~~~~~~~~~~~~~~~~~~~~~~~~~~~~~~~~~~~~~
%~~~~~~~~~~~~~~~~~~~~~~~~~~~~~~~~~~~~~~~~~~~~~~~~~~~~~~~~~~~~~~~~~~~
%~~~~~~~~~~~~~~~~~~~~~~~~~~~~~~~~~~~~~~~~~~~~~~~~~~~~~~~~~~~~~~~~~~~
%~~~~~~~~~~~~~~~~~~~~~~~~~~~~~~~~~~~~~~~~~~~~~~~~~~~~~~~~~~~~~~~~~~~
\begin{frame}
\centering
\Large

Challenges and limitations of machine learning
\end{frame}    
%~~~~~~~~~~~~~~~~~~~~~~~~~~~~~~~~~~~~~~~~~~~~~~~~~~~~~~~~~~~~~~~~~~~
\begin{frame}{Challenges of supervised learning}

    \only<1>{\bi
\item Over- and under-fitting of the training data
       \bi
   \item Algorithms do poorly on new/unseen data
       \ei

\ei

\centering
    \includegraphics[height=.5\textheight]{figs/overfitting.png}

    {\footnotesize\url{https://towardsdatascience.com/underfitting-and-overfitting-in-machine-learning-and-how-to-deal-with-it-6fe4a8a49dbf}}
    }

    \only<2>{\bi
   \item Many algorithms are considered ``black box'' algorithms
       \bi
   \item Lack interpretability (\textit{e.g.} What features are the most important for classification? What are the feature thresholds?)
       \ei

\ei

\centering
    \includegraphics[height=.5\textheight]{figs/blackbox.jpg}

    }

    \only<3>{\bi
   \item Many sophisticated algorithms are considered ``black box'' algorithms
       \bi
   \item Lack interpretability (\textit{e.g.} What features are the most important for classification? What are the feature thresholds?)
       \ei

\ei

\centering
    \includegraphics[height=.5\textheight]{figs/interpret_ml.jpg}

    {\hfill\alert{\href{https://christophm.github.io/interpretable-ml-book/index.html}{Click for free book!}}}

    }


\end{frame}    
%~~~~~~~~~~~~~~~~~~~~~~~~~~~~~~~~~~~~~~~~~~~~~~~~~~~~~~~~~~~~~~~~~~~
\begin{frame}{Challenges of unsupervised learning}
    \bi
\item Lack of ``true'' values; how do you know your clustering is right? Where do you ``cut'' your hierarchical clustering dendrogram? How do you know what \textit{k} to use in k-means?
    \bi
\item Silhouette/gap measurements help but they are not always perfect!
    \ei 
\item Algorithms that use random start points (\textit{e.g.} k-means) can give different results with the same data
    \ei
\end{frame}
%~~~~~~~~~~~~~~~~~~~~~~~~~~~~~~~~~~~~~~~~~~~~~~~~~~~~~~~~~~~~~~~~~~~
%~~~~~~~~~~~~~~~~~~~~~~~~~~~~~~~~~~~~~~~~~~~~~~~~~~~~~~~~~~~~~~~~~~~
\begin{frame}

\centering
\Large

How you can apply machine learning to your work

\end{frame}
%~~~~~~~~~~~~~~~~~~~~~~~~~~~~~~~~~~~~~~~~~~~~~~~~~~~~~~~~~~~~~~~~~~~
\begin{frame}{\normalsize Quality control; PCA, PLS-DA, hierarchical clustering}


\end{frame}
%~~~~~~~~~~~~~~~~~~~~~~~~~~~~~~~~~~~~~~~~~~~~~~~~~~~~~~~~~~~~~~~~~~~
\begin{frame}{PSM quality estimation; Percolator}


\end{frame}    
%~~~~~~~~~~~~~~~~~~~~~~~~~~~~~~~~~~~~~~~~~~~~~~~~~~~~~~~~~~~~~~~~~~~
\begin{frame}{-omics databases are resources for training data!}

Even though it may seem daunting... you can train your own supervised ML algorithm!

Do you ever have to sort through things manually and wish there was a tool that could just ``find'' things automatically? Maybe certain type of hard to find protein or predicted ORF? 

\end{frame}
%~~~~~~~~~~~~~~~~~~~~~~~~~~~~~~~~~~~~~~~~~~~~~~~~~~~~~~~~~~~~~~~~~~~
\begin{frame}{Resources to check out}
    
\be
\item Links in footnotes
\item \url{https://vas3k.com/blog/machine_learning/}
\item \url{http://faculty.marshall.usc.edu/gareth-james/}
\ee

\end{frame}    

\end{document}
